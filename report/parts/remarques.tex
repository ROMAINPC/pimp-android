\section{Remarques et améliorations  :}

\subsection{Remarques sur le code} \label{remarques_code}
A propos de la classe \textbf{Image} \ref{classeImage}, nous sauvegardons systématiquement une copie des pixels d'origine de l'Image, de plus si un appel à \textbf{quickSave()} est effectué une 3ème copie de l'Image est chargée en mémoire. Image offre cependant des constructeurs pour charger une image proportionnée à l'écran de l'appareil. Le risque de débordement mémoire est donc largement évité par cette limitation de taille.
\\

Lors du chargement d'une nouvelle image, nous ré-instancions un objet de la classe Image. Ce qui veut techniquement dire que jusqu'au prochain passage du ramasse miette Android, deux images sont en mémoires, donc deux Bitmap et deux tableaux de pixels (la copie originale des Images, voir \ref{classeImage}). C'est un élément discutable cependant notre application limite la taille des Images chargées. Ce qui évite largement les dépassements mémoire.
\\

Dans la partie \ref{navig} nous créons une \textbf{MainActivity} après avoir récupéré une \textbf{Uri}, il aurait été idéal de rester dans \textbf{FirstActivity} jusqu'à chargement complet de la première image. Ce qui éviterait un éventuel aller-retour entre les activités en cas d'erreur. Cependant les limites de \textbf{Parcelable} \ref{parcelable} ainsi que l'utilisation d'une \textbf{AsyncTask} (donc un Thread différent) nous ont contraint à garder ce fonctionnement.
\\

Comme dit précédemment nous chargeons une image plus petite que l'originale afin d'optimiser la fluidité de l'application, à l'export en revanche le fichier image d'origine est chargé pour subir la même suite d'effets. Cette méthode ne permet pas d'empêcher un crash mémoire si l'image est immense (image d'appareil photo par exemple). On pourrait à l'avenir complexifier l'export pour charger et sauvegarder par morceaux le fichier image. Cependant ce n'est pas une priorité, la majorité des utilisateurs de l'application vont éditer des images venant d'appareils mobiles qui dépassent rarement les 20Mpx.
Nous avons défini une limite de 5000x5000px comme taille d'image en entrée, les images au dessus de cette taille seront redimensionnées en facteur $x^2$ à la sauvegarde pour assurer la fluidité de l'application.
\\

Le fait de charger l'image d'origine pour refaire les effets a comme conséquence que certains effets, notamment ceux de convolution aient une différence entre l'aperçu et l'image sauvegardée\ref{limits_conv}. Une possible solution à ce problème pourrait être d'afficher ou travailler avec un "crop" de l'image d'origine pour ces effets là.
\\

\subsection{Remarques sur les librairies Android}
Lors de la construction des instances d'\textbf{Image}\ref{classeImage}, nous devons passer la référence de l'activité contextuelle à l'Image, bien que pas très intuitif cette référence est nécessaire car utilisée par les méthodes de \textbf{Bitmap} de chargement d'image.
\\

\label{parcelable}
Pour manipuler des objets d'une activité à l'autre ou entre fragments, Android utilise des \textbf{Intent} ou des \textbf{Bundle}, passer des objets entiers devient assez lourd dans le code et nécessite l'utilisation de l'interface \textbf{Parcelable}, de plus passer un objet trop gros entraîne une \textbf{RuntimeException}. Finalement au sein d'une même application on peut se demander s'il est bien nécessaire de systématiser leur utilisation ou s'il ne serait pas plus simple de passer une référence ou simplement faire des accès statiques (au risque de perdre un peu la modularité du code).

\subsection{Améliorations à court terme :}

L'utilisation d'un historique d'effet pourra aussi permettre à l'utilisateur de sauvegarder des séries d'effets pour créer des effets personnalisés, c'est d'ailleurs pourquoi la classe \textbf{ImageEffect} contient des chaînes de caractères dédiées à la description de l'effet en question (nom et paramètres).

\subsection{Tests et bugs repérés :}
Des nombreux tests ont été faits pour assurer la stabilité de l'application, comme charger une image, supprimer cette dernière du stockage et puis sauvegarder celle qui est chargée dans l'application, essayer de charger un fichier corrompu, etc. Ces tests nous ont permis de repérer et de régler plusieurs bugs d'entrée et sortie de fichiers. Cependant nous n'avons pas trouvé des bugs pour l'instant.


\section{Remarques et améliorations  :}

\subsection{Remarques sur le code}
A propos de la classe \textbf{Image} \ref{classeImage}, nous sauvegardons systématiquement une copie des pixels d'origine de l'Image, de plus si un appel à \textbf{quickSave()} est effectué une 3ème copie de l'Image est chargée en mémoire. Image offre cependant des constructeurs pour charger une image proportionnée à l'écran de l'appareil. Le risque de débordement mémoire est donc largement évité par cette limitation de taille.
\\

Lors du chargement d'une nouvelle image, nous ré-instancions un objet de la classe Image. Ce qui veut techniquement dire que jusqu'au prochain passage du ramasse miette Android, deux images sont en mémoires, donc deux Bitmap et deux tableaux de pixels (la copie originale des Images, voir \ref{classeImage}). C'est un élément discutable cependant notre application limite la taille des Images chargées. Ce qui évite largement les dépassements mémoire.
\\

Dans la partie \ref{navig} nous créons une \textbf{MainActivity} après avoir récupéré une \textbf{Uri}, il aurait été idéal de rester dans \textbf{FirstActivity} jusqu'à chargement complet de la première image. Ce qui éviterait un éventuel aller-retour entre les activités en cas d'erreur. Cependant les limites de \textbf{Parcelable} \ref{parcelable} ainsi que l'utilisation d'une \textbf{AsyncTask} (donc un Thread différent) nous ont contraint à garder ce fonctionnement.
\\

Comme dit précedemment nous chargeons une image plus petite que l'originale afin d'optimiser la fluidité de l'application, à l'export en revanche le fichier image d'origine est chargé pour subir la même suite d'effets. Cette méthode ne permet pas d'empêcher un crash mémoire si l'image est immense (image d'appareil photo par exemple). On pourrait à l'avenir complexifier l'export pour charger et sauvegarder par morceaux le fichier image. Cependant ce n'est pas une priorité, la majorité des utilisateurs de l'application vont éditer des images venant d'appareils mobiles qui dépassent rarement les 20Mpx.
\\

\subsection{Remarques sur les librairies Android}
Lors de la construction des instances d'\textbf{Image}\ref{classeImage}, nous devons passer la référence de l'activité contextuelle à l'Image, bien que pas très intuitif cette référence est nécessaire car utilisée par les méthodes de \textbf{Bitmap} de chargement d'image.
\\

\label{parcelable}
Pour manipuler des objets d'une activité à l'autre ou entre fragments, Android utilise des \textbf{Intent} ou des \textbf{Bundle}, passer des objets entiers devient assez lourd dans le code et nécessite l'utilisation de l'interface \textbf{Parcelable}, de plus passer un objet trop gros entraîne une \textbf{RuntimeException}. Finalement au sein d'une même application on peut se demander s'il est bien nécessaire de systématiser leur utilisation ou s'il ne serait pas plus simple de passer une référence ou simplement faire des accès statiques (au risque de perdre un peu la modularité du code).

\subsection{Améliorations à court terme :}
Il pourrait être intéressant de montrer des aperçus des effets en bas de l'écran dans la liste d'effets, pour cela il suffira de créer des instances d'Image générées à partir de l'image éditée mais en dimension inférieure. Les constructeurs sont déjà disponibles.
\\

Il peut être possible dans les classes d'effets de rajouter des surcharges aux méthodes d'effets pour leur passer directement une instance d'\textbf{Image}\ref{classeImage}. Il sera alors nécessaire de rajouter à Image une référence de son activité contexte, ce qui se fait facilement puisque l'image est chargée directement depuis MainActivity et y restera toujours.
\\
Cette simplification anodine offrirait un tout petit peu plus de lisibilité du code, cependant son ajout est débattable puisqu'elle rendrait les effets dépendants de la classe Image, ce qui romprait la modularité du code.
\\

Dans la classe Image nous utilisons une file d'effets \ref{file_effets}, pour l'instant cette file fournie par Image n'est pas vraiment reliée à la classe elle même dans le sens où l'utilisateur de la classe Image a le contrôle total de cette file.
\\
Il pourrait être tout à fait possible de masquer la gestion de cette file, qui se remplirait de manière automatique, ainsi il sera possible de faire une méthode \textbf{export()} dans la classe Image pour sauvegarder l'image, sans se soucier de gérer cette file. De même il serait possible d'annuler les derniers effets en ré appliquant sur l'image d'origine tout ceux que l'on souhaite conserver.
\\
Il faudrait cependant modifier le comportement de \textbf{quickSave()} et \textbf{discard()} afin de supprimer ou ré-ajouter la tête de file lors de l'aperçu des effets en glissant les curseurs.
\\
Cependant la principale problématique de cette fonctionnalité serait de passer d'un système où l'on appel les effets en passant l'Image (ou la Bitmap) en paramètre, à un système avec une méthode \textbf{applyEffect(...)} dans la classe Image. La question se pose alors de savoir comment préciser l'effet en question sans contraindre la classe Image à utiliser une batterie d'effets précise, ce qui romprait la modularité du code en rendant Image dépendant de nos classes d'effets. Ce problème est en réalité déjà réglé avec l'utilisations des \textbf{BitmapRunnable}, en effet il suffira de passer ces derniers à la méthode ce qui mettra une couche d'abstraction entre le terme "effet" et les classes de calculs des effets.
\section{Gestion du projet:}

\subsection{Organisation générale :}
Développer à plusieurs un code de taille moyenne sur une période de temps limitée nécessite une bonne organisation du projet pour limiter les conflits de code. En particulier sur un développement Android où des bugs propres à certains appareils apparaissent régulièrement.
\\

Nous avons évidemment travaillé avec git pour gérer notre code, un workflow utilisant des branches a été mis en place, même si finalement les branches ont été moins nombreuses et plus volumineuses que prévu, leur utilisation a permis le développement en parallèle de beaucoup de fonctionnalités. Les parties conçernant les effets, l'interface, la navigation et la structuration des classes ont ainsi pu être développées indépendamment.
\\
Certaines fonctionnalités dépendaient cependant du développement d'autres fonctionnalités, c'est pourquoi en plus du \textit{GitHub}, nous avions une liste de tâches à effectuer sur la plateforme \textit{Trello}. De plus nous communiquions et débâtions sur la structure du code régulièrement par messagerie.


\subsection{Avis personnels :}

"lalalalalala"
\\
\textit{Manuel Ricardo Guevara Garban}\\

"lalalalalala"
\\
\textit{Loïc Lachiver}\\

"lalalalalala"
\\
\textit{Romain Pigret-Cadou}\\

"lalalalalala"
\\
\textit{Sofiane Menadjlia}